\documentclass[12pt]{article}
\usepackage{cite}
\usepackage{amsmath}
\usepackage[a4paper, margin=0.8in, bottom=1.0in]{geometry}
\usepackage{graphicx}
\graphicspath{ {images/} }
\usepackage{listings}
\usepackage{textcomp}
\usepackage[utf8]{inputenc}
\usepackage[english]{babel}
\usepackage{color}

\definecolor{dkgreen}{rgb}{0,0.6,0}
\definecolor{gray}{rgb}{0.5,0.5,0.5}
\definecolor{mauve}{rgb}{0.58,0,0.82}

\lstset{frame=tb,
  language=Python,
  aboveskip=3mm,
  belowskip=3mm,
  showstringspaces=false,
  columns=flexible,
  keywordstyle=\color{blue},
  commentstyle=\color{dkgreen},
  basicstyle={\small\ttfamily},
  numbers=none,
  breaklines=true,
  breakatwhitespace=true,
  tabsize=3
}

\title{Cryptanalysis of Book-Caesar-RSA Cipher}

\author{Samanvitha Basole and Shubham Pachpute\\
\\
CS 265 Cryptography and Computer Security\\
\\
San Jose State University
}

\date{\today}

\begin{document}
\maketitle

% ~~~~~~~~~~~~~~~~~~~~~~~~ PROBLEM STATEMENT ~~~~~~~~~~~~~~~~~~~~~~~~
\section{Problem Statement}
From the Mystery Twister website, we attempt to solve the BCR code challenge. In this challenge, we help Alice and Bob find the treasure. Given the book text: \\
$'SIERRA-ZERO-JULIET-SIX-YANKEE-ONE-ROMEO-PAPPAEIGHT\\
-KILO-FIVE-UNIFORM-XRAY-XXX-BRAVO-VICTORTWO-FOUR-T\\
ANGO-MIKE-OSCAR-HOTEL-DELTA-QUEBECKFOXTROT-ALPHA\\
-YYY-LIMA-INDIA-THREE-WHISKEYNOVEMBER-ECHO\\
-CHARLIE-GOLF-ZULU', $\\
our task is to find the two words marked as XXX and YYY then use Book, Caesar, RSA ciphers to get the final output. 
% ~~~~~~~~~~~~~~~~~~~~~~~~ APPROACH ~~~~~~~~~~~~~~~~~~~~~~~~
\section{Approach}
In this paper, we solve the BCR cipher consisting of a three-stage cascade: Altered book, Caesar, and RSA. Each stage uses the output of the previous one as input as shown in Figure \ref{fig:arch}.
\begin{figure}[h]
  \includegraphics[scale=0.8]{3_stage}
  \caption{General Architecture}
  \label{fig:arch}
\end{figure}
We discuss the high-level algorithm design of Book, Caesar, and RSA ciphers, then we present the algorithm pseudocode and part of the implementation. Next, we explain our decryption process and analysis achieved through computational experiments. Finally, we provide screenshots of our solution and submission.


% ~~~~~~~~~~~~~~~~~~~~~~~~ ALGORITHM DISCUSSION ~~~~~~~~~~~~~~~~~~~~~~~~
\section{Algorithms Discussion}
\subsection{Book Cipher}
Book cipher uses letters of subsequent words in a text to encode messages. Essentially, these words form the key. Book cipher is a cipher where the algorithm uses a book or a piece of text for encryption. In this algorithm we replace the words in the plaintext with the location of the words in the book or text used for encryption. It is necessary to have the word in the plaintext present in the book to be used or else, sometimes it is also replaced by using letters in the words. The main idea of the book cipher is the key used for encryption. Both the parties exchanging the information should agree to use a book or a publication available to both of them which will be used as the key. The attacker intercepting the message should be able to guess the key to be able to decrypt the message. Widely used books are the Dictionary and the Bible.

\subsection{Caesar Cipher}
Caesar cipher is a simple substitution cipher in which the alphabet is shifted a certain number of characters based on the key. It is a symmetric key algorithm, that is, the same key is used for encryption and decryption \cite{busta2002encryption}. \\
We show the following example to demonstrate encryption and decryption using Caesar cipher: 
Given a Caesar cipher with a left shift of 3, we first convert letters to numbers:
\begin{lstlisting}
Letters: 	A   B   C   D   E   F   ...   X    Y    Z
Numbers: 	0   1   2   3   4   5   ...   23   24   25
\end{lstlisting}
To encrypt the plaintext 'ATTACK AT DAWN', we
\begin{itemize}
\item Transfom each letter in the plaintext into a number using the above scheme
\item Apply the rule: (number + key) mod 26 to each transformed number
\item Tranform each number back to its corresponding letter using the same scheme above
\end{itemize}
The result is the following ciphertext XQQXZH XQ AXTK. \\ 
Decryption is done in a similar way except the following rule is applied in the second step: (number - key) mod 26.

\subsection{RSA}
RSA is an algorithm that provides digital signature as well as encryption. RSA is based on factoring, that is, it is easy to factor two numbers but it is computationally NP-C to find two exact factors for a given number. This is a public key cryptosystem, and thus, the encryption key and the decryption key are different. \cite{rsaIntro}. \\
The RSA algorithm is as follows \cite{rsaSteps}:

\begin{itemize}
\item Select $p, q$ two large prime numbers
\item Calculate $N = pq$
\item Select integer $e$ relatively prime to $(p-1)(q-1)$
\item Calculate $d$ as the multiplicative inverse of $e$ modulo $(p-1)(q-1)$
\item Public key $(n, e)$
\item Private key $d$
\item Consider plaintext $M$
\item Convert plaintext to ciphertext by $C=M^{e}mod\ N$
\item Convert ciphertext to plaintext by $M=C^{d}mod\ N$
\end{itemize}

% ~~~~~~~~~~~~~~~~~~~~~~~~ ALGORITHM PSEUDOCODE ~~~~~~~~~~~~~~~~~~~~~~~~
\section{Algorithm Design}
\subsection{Book}
A part of the book cipher algorithm is listed below:
\begin{lstlisting}
xxx1 = "7"
yyy1 = "9"
for j in range(0, len(decodedText1)):
    # print "IN FOR 1\n"
    if decodedText1[i] == "XXX":
        decodedText1[i] = xxx1
    if decodedText1[i] == "YYY":
        decodedText1[i] = yyy1
\end{lstlisting}
\subsection{Caesar}
The hex input consists of letters and numbers while the key consists of numbers as offset. The below code demonstrates the main logic of the algorithm:
\begin{lstlisting}
output_str = ""
for eachChar in hex_string:
    try:
    	# if eachChar is a number
        digit = int(eachChar)
        output_str += str((digit - int(YY)) % 10)
    except:
    	# if eachChar is a letter
        output_str += letters[(letters.index(eachChar) + int(XX)) % 26]
\end{lstlisting}
\subsection{RSA}

% ~~~~~~~~~~~~~~~~~~~~~~~~ DECRYPTION ~~~~~~~~~~~~~~~~~~~~~~~~
\section{Decryption and Analysis}
...

% ~~~~~~~~~~~~~~~~~~~~~~~~ SUBMITTING SOLUTION ~~~~~~~~~~~~~~~~~~~~~~~~
\section{Solution Submission}
...

\bibliography{references}
\bibliographystyle{plain}

\end{document}